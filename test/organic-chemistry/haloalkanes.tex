\documentclass{article}
\usepackage{chemfig}
\usepackage{mhchem}
\title{Haloalkanes and Haloarenes}
\author{Abdullah Ahmad Siddiqui}
\date{24-02-2025}
\begin{document}
\maketitle
\tableofcontents 

\section{Introduction}
The replacement of hydrogen atoms by halogen atoms in hydrocarbons
yield haloalkanes and haloarenes.
Haloalkanes contain halogen atoms attached to $sp^3$
hybridized carbon atoms while haloarenes contain halogen
atoms attached to $sp^2$ hybridized atoms.
\subsection{Compounds containing $sp^3-x$ bond}
These include:
\begin{enumerate}
  \item Alkyl Halides or Haloalkanes (\chemfig{R-X})
  \item{Allylic Halides}
  \item{Benzylic Halides}
\end{enumerate}
\subsubsection{Alkyl Halides}
Here, the halogen is bonded to an alkyl group (R). They are further 
classified into primary, secondary and tertiary according to the
nature of carbon to which the halogen is attached.
\vspace{5mm}

\chemname{\chemfig{C(-[0]X)(-[2]H)(-[4]R)(-[6]H)}}{ Primary Halide }\qquad
\chemname{\chemfig{C(-[0]X)(-[2]R`)(-[4]R)(-[6]H)}}{ Secondary Halide} \qquad{}
\chemname{\chemfig{C(-[0]X)(-[2]R`)(-[4]R)(-[6]R``)}}{  Tertiary Halide}
\subsubsection{Allylic Halides}
Here, the halogen is bonded to an $sp^3$ hybridized carbon adjacent
to which is a carbon - carbon double bond. For example:
\vspace{5mm}

\chemfig{=(-[7]CH_2X)} 
\vspace{5mm}

\chemfig{*6(-(-X)-=---)}
\subsubsection{Benzylic Halides}
These are compounds where the $sp^3$ carbon attached to the halogen is bonded
to an aromatic ring.
\vspace{5mm}

\chemfig{*6(-=-(-CH_2X)=-=)}
\subsection{Compounds containing $sp^2-x$ bond}
These include
\begin{enumerate}
  \item Vinylic Halides
  \item{Aryl Halides}
\end{enumerate}
\subsection*{Vinylic Halides}
Here, the halogen is bonded to an $sp^2$ hybridized atom of a carbon - carbon
double bond. (\chemfig{C=C})
\vspace{5mm}

\chemfig{=-[7]X} \qquad{} \chemfig{*6(--(-X)=---)}
\subsection*{Aryl Halides}
Here, the halogen is bonded to an $sp^2$ hybridized carbon which
is part of an aromatic ring.
\vspace{5mm}

\chemfig{*6(-=(-X)-=-=)}

\section{Preparation of Haloalkanes}
\subsection{From Alcohols}
\begin{enumerate}
 \item \ce{R-OH + HCl ->[ZnCl2] R-Cl + H2O}
 \item \ce{R-OH + NaBr + H2SO4 -> R-Br+ NaHSO4+H2O}
 \item \ce{3R-OH + PX3 -> 3R-X +H3PO3}
 \item \ce{3R-OH + PCl5 -> 3R-Cl +HCl +POCl3}
 \item \ce{3R-OH ->[red P/X2][X2 = Br2,I2] R-X}
 \item \ce{3R-OH + SOCl2 -> R-Cl +SO2 +HCl}
\end{enumerate}
\subsection{Free Radical Halogenation}
\vspace{5mm}

\ce{\chemfig{-[1]-[7]-[1]-[7]-[1]} ->[Cl2 / UV Light][or heat] \chemfig{-[1]-[7]-[1]-[7]-[1]-[7]Cl} +  \chemfig{-[1]-[7]-[1]-[7](-[6]Cl)-[1]}}
\subsection{From Alkenes}
\subsubsection{Addition of Hydrogen Halides}
\vspace{5mm}
\ce{ \chemfig{(-[3])(-[5])=(-[1])(-[7])} +HX -> \chemfig{(-[3])(-[5])(-[6]H)-(-[1])(-[7])(-[6]X)}}
\subsubsection{Addition of Halogens}
\vspace{5mm}
\ce{ \chemfig{(-[3])(-[5])=(-[1])(-[7])} +X2 ->[CCl4] \chemname{ \chemfig{(-[3])(-[5])(-[6]X)-(-[1])(-[7])(-[6]X)}  }{vic-Dihalide}}
\subsection{Halogen Exchange}
These reactions proceed by the displacement of one halogen with another.
In protoc solvents, halogens with smaller hydrated size displace halogens with 
greater hydrated size, while in aprotic solvents, halogens with smaller atomic size
displace halogens with greater atomic size.
\subsubsection{In Protic Solvents}
\vspace{5mm}
\ce{R-X + NaX` -> R-X` + NaX}
\vspace{5mm}
This reaction is known as \textbf{Finkelstein Reaction}. Here halogen with 
bigger size displaces halogen with smaller size.
Alkyl iodides are often prepared by this reaction.
\subsubsection{In Aprotic Solvents}
\vspace{5mm}
\ce{R-X + NaX` -> R-X` + NaX}
\vspace{5mm}

This reaction is known as \textbf{Schwartz Reaction}. Here halogen with 
smaller size displaces halogen with bigger size. Alkyl fluorides are often 
prepared by this reaction.
\section{Preparation of Haloarenes}

\end{document}
