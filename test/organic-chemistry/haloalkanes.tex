\documentclass{article}
\usepackage{chemfig}
\title{Haloalkanes and Haloarenes}
\author{Abdullah Ahmad Siddiqui}
\date{24-02-2025}
\begin{document}
\maketitle
\tableofcontents 

\section{Introduction}
The replacement of hydrogen atoms by halogen atoms in hydrocarbons
yield haloalkanes and haloarenes.
Haloalkanes contain halogen atoms attached to $sp^3$
hybridized carbon atoms while haloarenes contain halogen
atoms attached to $sp^2$ hybridized atoms.
\subsection{Compounds containing $sp^3-x$ bond}
These include:
\begin{enumerate}
  \item Alkyl Halides or Haloalkanes (\chemfig{R-X})
  \item{Allylic Halides}
  \item{Benzylic Halides}
\end{enumerate}
\subsection{Alkyl Halides}
Here, the halogen is bonded to an alkyl group (R). They are further 
classified into primary, secondary and tertiary according to the
nature of carbon to which the halogen is attached.

\newpage{}
\chemname{\chemfig{C(-[0]X)(-[2]H)(-[4]R)(-[6]H)}}{ Primary Halide }\qquad
\chemname{\chemfig{C(-[0]X)(-[2]R`)(-[4]R)(-[6]H)}}{ Secondary Halide} \qquad{}
\chemname{\chemfig{C(-[0]X)(-[2]R`)(-[4]R)(-[6]R``)}}{  Tertiary Halide}

\section{Preparation}
  
\end{document}
